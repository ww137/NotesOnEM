\section{Gauge Invariance}
\label{sec:05a-04a}

고전역학에서 potential은 차이가 중요하여 임의의 상수를 더하거나 빼도 상관 없는 양이었다. 왜 four-potential이라는 이름의 의미를 알기 위해서 운동방정식 \eqref{eq:ll1705}를 먼저 관찰해보자. 운동방정식에서 전하의 운동에 영향을 미치는 것은 potential이 아니라 장의 세기인 $\vb{E}$와 $\vb{B}$임을 알 수 있다. 따라서 4-potential이 서로 같은 장에 대응된다면 물리적으로 동등하다.

만약