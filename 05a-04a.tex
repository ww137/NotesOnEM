\section{Gauge Invariance}
\label{sec:05a-04a}

고전역학에서 potential은 차이가 중요하여 임의의 상수를 더하거나 빼도 상관 없는 양이었다. Four-potential이라는 이름의 의미를 알기 위해서 운동방정식 \eqref{eq:ll1705}를 먼저 관찰해보자. 운동방정식에서 전하의 운동에 영향을 미치는 것은 potential이 아니라 장의 세기인 $\vb{E}$와 $\vb{B}$임을 알 수 있다. 따라서 4-potential이 서로 같은 장에 대응된다면 물리적으로 동등하다.

만약 potential $\phi$와 $\vb{A}$가 주어져 있다면 $\vb{E}$와 $\vb{B}$는 유일하게 결정된다. 그러나 하나의 전자기장에는 여러 potential이 대응된다. 이를 보이기 위해서 임의의 시공간 좌표에 대한 함수 $f$에 대하여 다음과 같은 변환을 생각해보자.
\begin{equation}\label{eq:ll1801}
    \tensor{A}{_\mu} \to \tensor{A}{_\mu} - \tensor{\partial}{_\mu} f
\end{equation}
이는 작용에 다음과 같은 항을 추가한다.
\begin{equation}
    q \; \dd \tensor{x}{^\mu} \; \tensor{\partial}{_\mu} f = d ( q f )
\end{equation}
그러나 이 항은 전미분이다. 따라서 운동방정식에 아무런 영향을 주지 않는다.

\eqref{eq:ll1801}은 다음과 같이 쓸 수도 있다.
\begin{equation}\label{eq:ll1803}
    \vb{A} \to \vb{A} + \grad f \qquad \phi \to \phi - \pdv{f}{t}
\end{equation}
\eqref{eq:ll1803}을 \eqref{eq:ll1703}\과 \eqref{eq:ll1704}에 적용해도 전기장과 자기장이 변하지 않음을 쉽게 확인할 수 있다. 따라서 \eqref{eq:ll1801}은 장을 바꾸지 않는다.

\eqref{eq:ll1803}의 변환에 대해서 변하지 않는 양만이 물리적인 의미가 있다. 특히 모든 방정식은 이러한 변환에 대하여 불변이어야만 한다. 이러한 불변성을 gauge 불변(gauge invariance)이라고 한다.

Potential이 유일하지 않기에 문제를 풀기에 좋은 조건을 설정하고 $f$를 적당히 잡아서 potential이 언제나 특정한 방정식을 만족하게 할 수 있다. 이러한 행위를 gauge 고정(gauge fixing)이라 한다.
