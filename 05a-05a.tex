\section[Constant EM Field]{Constant Electromagnetic Field}
\label{sec:05a-05a}

Constant electromagnetic field는 장이 시간에 의존하지 않는다는 뜻으로 정적인 전자기장으로 번역할 수 있다. 분명히 정적인 장의 potential은 시간에 의존하지 않도록 잡을 수 있다. 이 경우에 다음이 성립한다.
\begin{equation}\label{eq:ll1901}
    \vb{E} = - \grad \phi \qquad \vb{B} = \curl \vb{A}
\end{equation}
따라서 전기장과 자기장은 각각 scalar potentail과 vector potential로 결정되며 서로 얽혀있지 않게\footnote{Decoupled} 된다.

Section \ref{sec:05a-04a}에서 potential은 유일하게 결정되지 않는다는 것을 보았다. 하지만 정적인 전자기장을 시간에 의존하지 않는 potential로 나타낸다면 scalar potential의 경우에는 오직 상수\footnote{시공간 좌표에 의존하지 않는}만을 더할 수 있다. 보통 이러한 경우 $\phi$는 공간상의 특정한 점에서 특정한 값을 갖도록 설정한다. 사실 $\lim _{r \to \infty} \phi = 0$이도록 잡는 것이 더 흔하다. 이런 방식으로 정적인 장의 scalar potential을 유일하게 결정한다.

한편 vector potential은 정적인 장에 대해서도 유일하게 결정되지 않는다. 언제나 (시간에 의존하지 않고) 공간 좌표에 의존하는 임의의 scalar 함수의 gradient를 더할 수 있다.

이제 정적인 전자기장 안의 전하의 energy를 구해보자. 장이 정적이면 전하의 Lagrangian또한 시간에 대하여 명시적으로 의존하지 않는다. 이러한 경우에 energy는 보존되며 Hamiltonian과 같다.

\eqref{eq:ll1606}에서 다음을 얻는다.
\begin{equation}\label{eq:ll1902}
    \mathcal{E} = \frac{m c^2}{\sqrt{1 - \frac{v^2}{c^2}}} + q \phi
\end{equation}
이로부터 장이 존재하면 입자의 energy에 $q \phi$이 더해지는 것을 알 수 있다. 이것이 입자의 potential energy이다. 그런데 $\vb{A}$는 어디에 있는가? 이는 자기장이 전하의 energy에 영향을 주지 않는다는 것을 의미한다. 오직 전기장만이 입자의 energy를 변하게 한다. 이것은 자기장이 전기장과는 다르게 전하에 일을 하지 않는 것과 관련있다.

장의 세기가 공간상의 모든 점에서 같다면 장이 균일(uniform)하다고 한다.
균일한 전기장의 scalar potential은 다음과 같이 나타낼 수 있다.
\begin{equation}\label{eq:ll1903}
    \phi = - \vb{E} \cdot \vb{r}
\end{equation}
사실 $ \grad(\vb{E \cdot \vb{r}}) = ( \vb{E} \cdot \grad ) \vb{r} = \vb{E} $이다.

균일한 자기장의 vector potential은 다음과 같이 나타낼 수 있다.
\begin{equation}\label{eq:ll1904}
    \vb{A} = \frac{1}{2} ( \vb{B} \times \vb{r} )
\end{equation}
$ \curl (\vb{B} \times \vb{r}) = \vb{B} (\div \vb{r}) - ( \vb{B} \cdot \grad) \vb{r} = 2 \vb{B} $이고 $ 
\div \vb{r} = 3$이다.

균일한 자기장의 vector potential은 다음과 같이 쓸 수도 있다.
\begin{equation}\label{eq:ll1905}
    A_x = - B y \qquad A_y = 0 = A_z
\end{equation}
\eqref{eq:ll1904}와 \eqref{eq:ll1905}는 모두 동일한 자기장을 나타낸다. 사실 \eqref{eq:ll1904}에 $ f = -x y B / 2 $로 gague 변환\footnote{$ \grad f $를 더하면}을 하면 \eqref{eq:ll1905}를 얻는다.
