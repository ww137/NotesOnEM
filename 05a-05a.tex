\section[Constant EM Field]{Constant Electromagnetic Field}
\label{sec:05a-05a}

Constant electromagnetic field는 장이 시간에 의존하지 않는다는 뜻으로 정적인 전자기장으로 번역할 수 있다. 분명히 정적인 장의 potential은 시간에 의존하지 않도록 잡을 수 있다. 이 경우에 다음이 성립한다.
\begin{equation}\label{eq:ll1901}
    \vb{E} = - \grad \phi \qquad \vb{B} = \curl \vb{A}
\end{equation}
따라서 전기장과 자기장은 각각 scalar potentail과 vector potential로 결정되며 서로 얽혀있지 않게\footnote{Decoupled} 된다.

Section \ref{sec:05a-04a}에서 potential은 유일하게 결정되지 않는다는 것을 보았다. 하지만 정적인 전자기장을 시간에 의존하지 않는 potential로 나타낸다면 scalar potential의 경우에는 오직 상수\footnote{시공간 좌표에 의존하지 않는}만을 더할 수 있다. 보통 이러한 경우 $\phi$는 공간상의 특정한 점에서 특정한 값을 갖도록 설정한다. 사실 $\lim _{r \to \infty} \phi = 0$이도록 잡는 것이 더 흔하다. 이런 방식으로 정적인 장의 scalar potential을 유일하게 결정한다.

한편 vector potential은 정적인 장에 대해서도 유일하게 결정되지 않는다. 언제나 (시간에 의존하지 않고) 공간 좌표에 의존하는 임의의 scalar 함수의 gradient를 더할 수 있다.
